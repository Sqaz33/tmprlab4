\documentclass{article}
\usepackage[russian]{babel}
\usepackage[T2A]{fontenc}
\usepackage[utf8]{inputenc}
\usepackage{amsmath}
\usepackage{geometry}
\geometry{a4paper, left=20mm, right=20mm, top=20mm, bottom=20mm}

\begin{document}

\section{Математическая модель транспортной задачи}

\subsection{Переменные}
Пусть \( x_{ij} \) — количество груза, перевозимого из пункта поставки \( A_i \) в пункт потребления \( B_j \), где \( i = 1, 2, 3 \), \( j = 1, 2, 3, 4, 5 \).

\subsection{Ограничения}
1. По запасам:
\[
\begin{cases}
x_{11} + x_{12} + x_{13} + x_{14} + x_{15} = 200, \\
x_{21} + x_{22} + x_{23} + x_{24} + x_{25} = 250, \\
x_{31} + x_{32} + x_{33} + x_{34} + x_{35} = 160.
\end{cases}
\]

2. По потребностям:
\[
\begin{cases}
x_{11} + x_{21} + x_{31} = 120, \\
x_{12} + x_{22} + x_{32} = 120, \\
x_{13} + x_{23} + x_{33} = 100, \\
x_{14} + x_{24} + x_{34} = 210, \\
x_{15} + x_{25} + x_{35} = 60.
\end{cases}
\]

3. Неотрицательность:
\[
x_{ij} \geq 0 \quad \forall i, j.
\]

\subsection{Целевая функция}
Минимизировать общие затраты на перевозку:
\begin{multline}
Z = 10x_{11} + 15x_{12} + 16x_{13} + 12x_{14} + 20x_{15} \\
+ 21x_{21} + 9x_{22} + 10x_{23} + 9x_{24} + 7x_{25} \\
+ 12x_{31} + 15x_{32} + 16x_{33} + 13x_{34} + 21x_{35} \to \min.
\end{multline}

\section{Метод северо-западного угла}

\subsection{Шаги}
1. Начинаем с верхнего левого угла матрицы (клетка \( x_{11} \)).  
2. Распределяем груз, пока не исчерпаем запас или потребность.

\subsection{Распределение}
\begin{enumerate}
\item \( x_{11} = \min(200, 120) = 120 \).  
Остаток в \( A_1 \): \( 200 - 120 = 80 \).  
Потребность \( B_1 \) удовлетворена.

\item \( x_{12} = \min(80, 120) = 80 \).  
Остаток в \( A_1 \): 0.  
Потребность \( B_2 \): \( 120 - 80 = 40 \).

\item \( x_{22} = \min(250, 40) = 40 \).  
Остаток в \( A_2 \): \( 250 - 40 = 210 \).  
Потребность \( B_2 \) удовлетворена.

\item \( x_{23} = \min(210, 100) = 100 \).  
Остаток в \( A_2 \): \( 210 - 100 = 110 \).  
Потребность \( B_3 \) удовлетворена.

\item \( x_{24} = \min(110, 210) = 110 \).  
Остаток в \( A_2 \): 0.  
Потребность \( B_4 \): \( 210 - 110 = 100 \).

\item \( x_{34} = \min(160, 100) = 100 \).  
Остаток в \( A_3 \): \( 160 - 100 = 60 \).  
Потребность \( B_4 \) удовлетворена.

\item \( x_{35} = \min(60, 60) = 60 \).  
Остаток в \( A_3 \): 0.  
Потребность \( B_5 \) удовлетворена.
\end{enumerate}

\subsection{Опорный план}
\[
\begin{array}{|c|c|c|c|c|c|}
\hline
 & B_1 & B_2 & B_3 & B_4 & B_5 \\
\hline
A_1 & 120 & 80 & 0 & 0 & 0 \\
\hline
A_2 & 0 & 40 & 100 & 110 & 0 \\
\hline
A_3 & 0 & 0 & 0 & 100 & 60 \\
\hline
\end{array}
\]

\subsection{Стоимость}
\begin{multline}
Z = 120 \cdot 10 + 80 \cdot 15 + 40 \cdot 9 + 100 \cdot 10 \\
+ 110 \cdot 9 + 100 \cdot 13 + 60 \cdot 21 \\
= 1200 + 1200 + 360 + 1000 \\
+ 990 + 1300 + 1260 = 7310.
\end{multline}

\section{Метод минимальной стоимости}

\subsection{Шаги}
1. На каждом шаге выбираем клетку с минимальной стоимостью.  
2. Распределяем груз, пока не исчерпаем запас или потребность.

\subsection{Распределение}
\begin{enumerate}
\item Минимальная стоимость \( c_{25} = 7 \).  
\( x_{25} = \min(250, 60) = 60 \).  
Остаток в \( A_2 \): \( 250 - 60 = 190 \).  
Потребность \( B_5 \) удовлетворена.

\item Следующая минимальная стоимость \( c_{24} = 9 \).  
\( x_{24} = \min(190, 210) = 190 \).  
Остаток в \( A_2 \): 0.  
Потребность \( B_4 \): \( 210 - 190 = 20 \).

\item Следующая минимальная стоимость \( c_{11} = 10 \).  
\( x_{11} = \min(200, 120) = 120 \).  
Остаток в \( A_1 \): \( 200 - 120 = 80 \).  
Потребность \( B_1 \) удовлетворена.

\item Следующая минимальная стоимость \( c_{14} = 12 \).  
\( x_{14} = \min(80, 20) = 20 \).  
Остаток в \( A_1 \): \( 80 - 20 = 60 \).  
Потребность \( B_4 \) удовлетворена.

\item Следующая минимальная стоимость \( c_{12} = 15 \).  
\( x_{12} = \min(60, 120) = 60 \).  
Остаток в \( A_1 \): 0.  
Потребность \( B_2 \): \( 120 - 60 = 60 \).

\item Следующая минимальная стоимость \( c_{32} = 15 \).  
\( x_{32} = \min(160, 60) = 60 \).  
Остаток в \( A_3 \): \( 160 - 60 = 100 \).  
Потребность \( B_2 \) удовлетворена.

\item Следующая минимальная стоимость \( c_{33} = 16 \).  
\( x_{33} = \min(100, 100) = 100 \).  
Остаток в \( A_3 \): 0.  
Потребность \( B_3 \) удовлетворена.
\end{enumerate}

\subsection{Опорный план}
\[
\begin{array}{|c|c|c|c|c|c|}
\hline
 & B_1 & B_2 & B_3 & B_4 & B_5 \\
\hline
A_1 & 120 & 60 & 0 & 20 & 0 \\
\hline
A_2 & 0 & 0 & 0 & 190 & 60 \\
\hline
A_3 & 0 & 60 & 100 & 0 & 0 \\
\hline
\end{array}
\]

\subsection{Стоимость}
\begin{multline}
Z = 120 \cdot 10 + 60 \cdot 15 + 20 \cdot 12 + 190 \cdot 9 \\
+ 60 \cdot 7 + 60 \cdot 15 + 100 \cdot 16 \\
= 1200 + 900 + 240 + 1710 \\
+ 420 + 900 + 1600 = 6970.
\end{multline}

\section{Метод Фогеля}

\subsection{Шаги}
1. Для каждой строки и столбца вычисляем разницу между двумя минимальными стоимостями.  
2. Выбираем строку или столбец с максимальной разницей.  
3. В выбранной строке/столбце распределяем груз в клетку с минимальной стоимостью.

\subsection{Итерации}
\begin{enumerate}
\item Разницы по строкам:  
\( A_1 \): 12 - 10 = 2,  
\( A_2 \): 9 - 7 = 2,  
\( A_3 \): 13 - 12 = 1.  

Разницы по столбцам:  
\( B_1 \): 12 - 10 = 2,  
\( B_2 \): 15 - 9 = 6,  
\( B_3 \): 16 - 10 = 6,  
\( B_4 \): 12 - 9 = 3,  
\( B_5 \): 7 - 7 = 0.  

Максимальная разница в столбцах \( B_2 \) и \( B_3 \) (6). Выбираем \( B_2 \), минимальная стоимость \( c_{22} = 9 \).  
\( x_{22} = \min(250, 120) = 120 \).  
Остаток в \( A_2 \): \( 250 - 120 = 130 \).  
Потребность \( B_2 \) удовлетворена.

\item Удаляем столбец \( B_2 \).  
Разницы по строкам:  
\( A_1 \): 12 - 10 = 2,  
\( A_2 \): 9 - 7 = 2,  
\( A_3 \): 13 - 12 = 1.  

Разницы по столбцам:  
\( B_1 \): 12 - 10 = 2,  
\( B_3 \): 10 - 10 = 0,  
\( B_4 \): 12 - 9 = 3,  
\( B_5 \): 7 - 7 = 0.  

Максимальная разница в столбце \( B_4 \) (3). Минимальная стоимость \( c_{24} = 9 \).  
\( x_{24} = \min(130, 210) = 130 \).  
Остаток в \( A_2 \): 0.  
Потребность \( B_4 \): \( 210 - 130 = 80 \).

\item Удаляем строку \( A_2 \).  
Разницы по строкам:  
\( A_1 \): 12 - 10 = 2,  
\( A_3 \): 13 - 12 = 1.  

Разницы по столбцам:  
\( B_1 \): 12 - 10 = 2,  
\( B_3 \): 16 - 16 = 0,  
\( B_4 \): 12 - 12 = 0,  
\( B_5 \): 21 - 20 = 1.  

Максимальная разница в строках \( A_1 \) (2) и столбцах \( B_1 \) (2). Выбираем \( A_1 \), минимальная стоимость \( c_{11} = 10 \).  
\( x_{11} = \min(200, 120) = 120 \).  
Остаток в \( A_1 \): \( 200 - 120 = 80 \).  
Потребность \( B_1 \) удовлетворена.

\item Удаляем столбец \( B_1 \).  
Разницы по строкам:  
\( A_1 \): 12 - 12 = 0,  
\( A_3 \): 13 - 13 = 0.  

Разницы по столбцам:  
\( B_3 \): 16 - 16 = 0,  
\( B_4 \): 12 - 12 = 0,  
\( B_5 \): 21 - 20 = 1.  

Все разницы нулевые, распределяем по минимальным стоимостям.  
\( x_{14} = \min(80, 80) = 80 \).  
Остаток в \( A_1 \): 0.  
Потребность \( B_4 \) удовлетворена.

\item Остается \( A_3 \).  
\( x_{33} = 100 \), \( x_{35} = 60 \).
\end{enumerate}

\subsection{Опорный план}
\[
\begin{array}{|c|c|c|c|c|c|}
\hline
 & B_1 & B_2 & B_3 & B_4 & B_5 \\
\hline
A_1 & 120 & 0 & 0 & 80 & 0 \\
\hline
A_2 & 0 & 120 & 0 & 130 & 0 \\
\hline
A_3 & 0 & 0 & 100 & 0 & 60 \\
\hline
\end{array}
\]

\subsection{Стоимость}
\begin{multline}
Z = 120 \cdot 10 + 80 \cdot 12 + 120 \cdot 9 \\
+ 130 \cdot 9 + 100 \cdot 16 + 60 \cdot 21 \\
= 1200 + 960 + 1080 + 1170 \\
+ 1600 + 1260 = 7270.
\end{multline}

\section{Проверка на оптимальность методом потенциалов}

Проверим план, полученный методом минимальной стоимости (он лучше других).

\subsection{Базисные клетки}
\( x_{11}, x_{12}, x_{14}, x_{24}, x_{25}, x_{32}, x_{33} \).

\subsection{Система потенциалов}
\[
\begin{aligned}
u_1 + v_1 &= 10, & u_1 + v_2 &= 15, & u_1 + v_4 &= 12, \\
u_2 + v_4 &= 9, & u_2 + v_5 &= 7, \\
u_3 + v_2 &= 15, & u_3 + v_3 &= 16.
\end{aligned}
\]

Полагаем \( u_1 = 0 \):
\[
\begin{aligned}
v_1 &= 10, & v_2 &= 15, & v_4 &= 12, \\
u_2 &= 9 - 12 = -3, & v_5 &= 7 - (-3) = 10, \\
u_3 &= 15 - 15 = 0, & v_3 &= 16 - 0 = 16.
\end{aligned}
\]

\subsection{Проверка свободных клеток}
\[
\Delta_{ij} = c_{ij} - (u_i + v_j).
\]

\begin{itemize}
\item \( \Delta_{13} = 16 - (0 + 16) = 0 \),
\item \( \Delta_{15} = 20 - (0 + 10) = 10 \),
\item \( \Delta_{21} = 21 - (-3 + 10) = 14 \),
\item \( \Delta_{22} = 9 - (-3 + 15) = -3 \) (отрицательная оценка).
\end{itemize}

План не оптимален. Улучшаем по клетке \( x_{22} \).

\subsection{Цикл пересчета}
\( x_{22} \rightarrow x_{12} \rightarrow x_{14} \rightarrow x_{24} \).  
Минимальный груз в минусовых клетках: \( \theta = \min(60, 190) = 60 \).  
Новый план:
\[
x_{22} = 60, \quad x_{12} = 0, \quad x_{14} = 80, \quad x_{24} = 130.
\]

\subsection{Обновленный план}
\[
\begin{array}{|c|c|c|c|c|c|}
\hline
 & B_1 & B_2 & B_3 & B_4 & B_5 \\
\hline
A_1 & 120 & 0 & 0 & 80 & 0 \\
\hline
A_2 & 0 & 60 & 0 & 130 & 60 \\
\hline
A_3 & 0 & 60 & 100 & 0 & 0 \\
\hline
\end{array}
\]

Стоимость: \( Z = 6970 - 3 \cdot 60 = 6790 \).

\section{Итоговый оптимальный план}

\[
\begin{array}{|c|c|c|c|c|c|}
\hline
 & B_1 & B_2 & B_3 & B_4 & B_5 \\
\hline
A_1 & 120 & 0 & 0 & 80 & 0 \\
\hline
A_2 & 0 & 60 & 0 & 130 & 60 \\
\hline
A_3 & 0 & 60 & 100 & 0 & 0 \\
\hline
\end{array}
\]

Минимальная стоимость: \( Z = 6790 \).

\end{document}